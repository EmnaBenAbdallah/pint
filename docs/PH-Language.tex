\documentclass[11pt]{article}
\usepackage{fullpage}
\usepackage{amsmath}
\usepackage{amssymb}
\usepackage{color}

\title{The Process Hitting Language}
\author{Lo\"ic Paulev\'e, Morgan Magnin, Olivier Roux}

\newcommand{\term}[1]{\text{\color{blue}\textbf{#1}}}
\newcommand{\comment}[1]{\hspace{1cm}\text{#1}}
\newcommand{\regexp}[1]{{\color{red}#1}}
\newcommand{\ch}[1]{{\color{black}#1}}

\newcommand{\hit}[3]{\mbox{$#1\rightarrow#2\Rsh#3$}}


\begin{document}
\maketitle

\section{Introduction}

\subsection{Directives}

Located at the top of the source code, two directives may be specified :
\begin{description}
\item[sample] Duration of the simulation.
\item[stochasticity\_absorption] Default stochasticity absorption factor. If not specified, the default is $1$ (no absorption).
\end{description}

\subsection{Process Definition}

Processes are defined by a name and a maximum level.

\noindent The follwing example defines a process $a$ with 2 levels $\{0,1\}$ and a process $b$ with 3 levels $\{0,1,2\}$:
\begin{verbatim}
process a 1
process b 2
\end{verbatim}


\subsection{Actions}

Declare actions $\hit{a_i}{b_j}{b_k}$ and attach to them a use rate and optionaly a stochasticity absorption factor.
If no stochasticity absorption is specified, the default stochasticity absorption factor is used.
Processes have to be defined before creating such an action.

\noindent The following example adds an action $\hit{a_1}{b_2}{b_0}$ with a use rate $r=0.5$ and a stochasticity absorption $sa = 10$:
\begin{verbatim}
a1 -> b2 0 @ 0.5 ~ 10
\end{verbatim}


\subsection{Initial state}

The initial state is by default all processes at level 0.
It may be changed through the statement \verb|initial_state| by giving a list of process level present initially.

\begin{verbatim}
initial_state a1,b2
\end{verbatim}


\newpage
\section{Language summary}

$\{ Definition \}$ stands for optional definitions.

\begin{eqnarray*}
Program &::=& \{ Directive_1 \dots Directive_M \}\\
&& Definition \\
&& Declaration_1 \dots Declaration_N\\
&& \{ \term{initial\_state}~Process_1\term{,}\dots\term{,} Process_K \}\\
\\
Directive &::=& \term{directive}~\term{sample}~Float \\
&& \term{directive}~\term{stochasticity\_absorption}~Integer\\
\\
Process &::=& Name~Integer\\
\\
Definition &::=& \term{process}~Process\\
\\
Declaration &::=& Definition \\
& | & Action\\
\\
Action &::=& Process~\term{$->$}~Process~Integer~\{\term{@}~Float~\{\term{\textasciitilde}~Integer\}\}\\
\\
Name & ::= & \regexp{(\ch{A}\cdots \ch{z})+}\\
Integer & ::= & \regexp{(\ch{0}\cdots \ch{9})+}\\
Float & ::= & \regexp{(\ch{0}\cdots \ch{9})+ \ch{.} (\ch{0}\cdots \ch{9})*}\\
\end{eqnarray*}

Comments are delimited by \term{(*} and \term{*)} and can be nested.


\end{document}
