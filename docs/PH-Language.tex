\documentclass[11pt]{article}
\usepackage{fullpage}
\usepackage{amsmath}
\usepackage{color}

\title{The Process Hitting Language}
\author{Lo\"ic Paulev\'e, Morgan Magnin, Olivier Roux}

\newcommand{\term}[1]{\text{\color{blue}\textbf{#1}}}
\newcommand{\comment}[1]{\hspace{1cm}\text{#1}}
\newcommand{\regexp}[1]{{\color{red}#1}}
\newcommand{\ch}[1]{{\color{black}#1}}


\begin{document}
\maketitle

\section{Introduction}

\newpage
\section{Language summary}

$\{ Definition \}$ stands for optional definitions.

\begin{eqnarray*}
Program &::=& \{ Directive_1 \dots Directive_M \}\\
&& Definition \\
&& Declaration_1 \dots Declaration_N\\
&& \{ \term{initial\_state}~Process_1\term{,}\dots\term{,} Process_K \}\\
\\
Directive &::=& \term{directive}~\term{sample}~Float \\
&& \term{directive}~\term{stochasticity\_absorption}~Integer\\
\\
Process &::=& Name~Integer\\
\\
Definition &::=& \term{process}~Process\\
\\
Declaration &::=& Definition \\
& | & Action\\
\\
Action &::=& Process~\term{$->$}~Process~Integer~\{\term{@}~Float~\{\term{\textasciitilde}~Integer\}\}\\
\\
Name & ::= & \regexp{(\ch{A}\cdots \ch{z})+}\\
Integer & ::= & \regexp{(\ch{0}\cdots \ch{9})+}\\
Float & ::= & \regexp{(\ch{0}\cdots \ch{9})+ \ch{.} (\ch{0}\cdots \ch{9})*}\\
\end{eqnarray*}

Comments are delimited by \term{(*} and \term{*)} and can be nested.


\end{document}
